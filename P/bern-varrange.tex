\documentclass[a4paper,12pt]{article}

%%% Packages %%%
\usepackage{fullpage}
\usepackage{amsmath}
\usepackage{amssymb}
\usepackage{gensymb}
\usepackage{url}
\usepackage{csquotes}
\usepackage{enumitem}
\usepackage{setspace}
\usepackage[utf8]{inputenc}
\usepackage[bottom,hang,flushmargin]{footmisc}

%%% Settings %%%
\setlength{\parindent}{0pt}
\frenchspacing
\urlstyle{same}
\MakeOuterQuote{"}
\setlist{nolistsep}
\setlist[itemize]{leftmargin=*}
\setlist[enumerate]{leftmargin=*}


\begin{document}


**Theorem:** Let $X$ be a [random variable](/D/rvar) following a [Bernoulli distribution](/D/bern):

$$ \label{eq:bern}
X \sim \mathrm{Bern}(p) \; .
$$

Then, the [variance](/D/mean) of $X$ is necessarily between 0 and 1/4:

$$ \label{eq:bern-var-range}
0 \leq \mathrm{Var}(X) \leq \frac{1}{4} \; .
$$


**Proof:** The [variance of a Bernoulli random variable](/P/bern-var) is

$$ \label{eq:bern-var}
X \sim \mathrm{Bern}(p) \quad \Rightarrow \quad \mathrm{Var}(X) = p \, (1-p)
$$

which can also be understood as a function of the [success probability](/D/bern) $p$:

$$ \label{eq:bern-var-p}
\mathrm{Var}(X) = \mathrm{Var}(p) = -p^2 + p \; .
$$

The first derivative of this function is

$$ \label{eq:dVar-dp}
\frac{\mathrm{d}\mathrm{Var}(p)}{\mathrm{d}p} = -2 \, p + 1
$$

and setting this deriative to zero

\begin{equation} \label{eq:dVar-dp-0}
\begin{split}
\frac{\mathrm{d}\mathrm{Var}(p_M)}{\mathrm{d}p} &= 0 \\
0 &= -2 \, p_M + 1 \\
p_M &= \frac{1}{2} \; ,
\end{split}
\end{equation}

we obtain the maximum possible variance

$$ \label{eq:bern-var-max}
\mathrm{max}\left[\mathrm{Var}(X)\right] = \mathrm{Var}(p_M) = -\left( \frac{1}{2} \right)^2 + \frac{1}{2} = \frac{1}{4} \; .
$$

The function $\mathrm{Var}(p)$ is monotonically increasing for $0 < p < p_M$ as $\mathrm{d}\mathrm{Var}(p)/\mathrm{d}p > 0$ in this interval and it is monotonically decreasing for $p_M < p < 1$ as $\mathrm{d}\mathrm{Var}(p)/\mathrm{d}p < 0$ in this interval. Moreover, as [variance is always non-negative](/P/var-nonneg), the minimum variance is

$$ \label{eq:bern-var-min}
\mathrm{min}\left[\mathrm{Var}(X)\right] = \mathrm{Var}(0) = \mathrm{Var}(1) = 0 \; .
$$

Thus, we have:

$$ \label{eq:bern-var-int}
\mathrm{Var}(p) \in \left[ 0, \; \frac{1}{4} \right] \; .
$$


\end{document}